In this article we explore the connection between epistemic logic and game theory. In games of imperfect information, this connection arises naturally: The players need to reason about the knowledge of their teammates and opponents and epistemic logic seems like a natural choice.

However, even in games of perfect information, there is also a connection to epistemic logic: In game theory, the standard assumption is that all players are rational and -- more crucially -- that this is \emph{common knowledge} among all players. Is this assumption valid and can it be formalized? What are the implications of this assumption? Specifically, does backward induction actually follow from this assumption? These are questions of epistemic nature and thus, they should be discussed using the terminology of epistemic logic.

\subsubsection*{Overview of this article}

The structure of this article is as follows: In \secref{sec:imperfect-information}, we discuss games of imperfect information and in \secref{sec:rationality} we consider the questions regarding common knowledge of rationality. Also, we define possible worlds models and epistemic logic as in \cite{fagin1995a}.