We set out to explore how game theory was connected to epistemic logic. We interpreted and explained the results from a number of sources, in order to present an array of tools and formalizations, which can be used to model and analyze games of high complexity. We discussed the fundamental topic of rationality in games and its relationship with the validity of backward induction.

From the perspective of an epistemic logician the concept of lies is a strange one, and to model or even understand it correctly can be a challenge. By looking at a simplified version, of the popular game Liar's Dice, we showed that the presented tools can be used to model games with not only imperfect information, but with lies as a core element of the game.