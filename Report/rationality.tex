In game theory, the default assumption is that all players are perfectly rational and that this is common knowledge between all players. From this assumption, often without additional arguments, game theorists assume that \emph{backward induction} can be used to determine outcomes and strategies of perfect-information games. Backward induction is an algorithm that computes the outcome of a game assuming that all players want to maximize their payoffs from all states of the game.

In this section we shall investigate what rationality is. We shall also examine why common knowledge of rationality implies correctness of backward induction.

\paragraph*{Backward induction}
The idea of backward induction is straightforward: When a game tree has been constructed and payoffs have been assigned to the leaves, use a bottom-up approach to determine the \emph{inductive outcome}, which can then be used to determine strategies. 

Starting in the last stages (the nodes before the leaves) of the game tree, determine the action which yields maximal payoff and save the payoffs in the current node, so that the maximal payoffs propagate upwards in the tree until the root is encountered.

\subsection{Definitions and Assumptions}
In order to talk about rationality in perfect-information games, some important definitions and assumptions must be presented. Note that the Liar's Dice example is not used in this section, as it is not a game of perfect information.

\paragraph*{Strategy} A player participating in a game chooses a strategy, as defined previously in \secref{seq:perfect-information}. This involves that a player, when choosing a strategy and deciding what action to take at some node $v$, act as though $v$ has been reached.

\paragraph*{Rationality} Rationality can now be defined as \textit{choosing an action which yields maximum payoff at any node $v$}. In other word, a player will never choose a strategy that results in less payoff than another strategy, independent of which node he/she resides in. 
It is important to understand that, even if $v$ is not reached, a player must still act rationally in $v$ for backward induction to work. This is also linked to the definition of strategy, as the player acts as though $v$ has been reached.

\paragraph*{Common knowledge of Rationality} The simple definition of rationality is enough, to show that backward induction works. Therefore the notion of common knowledge in epistemic logic is used. 
Essentially common knowledge of rationality boils down to all players being rational and all players knowing this fact. This piece of information is also known to all players, which results in an infinite chain of knowledge, which defines common knowledge. 

\subsection{Results}
As stated before, it's important to assume that common knowledge of rationality holds, in order for backward induction to work.
Intuitively, this requirement makes perfect sense, which is illustrated in \figref{prg:lec5}:
Assume that the black nodes in the game tree is where the player Ann has to select an action. She could either end the game by taking action 1, and receive 3 utils as payoff, or continue the game by choosing action 2. By backward induction it is obvious that she should end the game by choosing action 1, because she knows that Bob is rational and will choose payoffs $A:2$, $E:4$. By choosing action 2, Ann would not optimize her payoff, because she would only receive 2 utils. Let's say Ann doesn't know that Bob will act rationally. If Ann considers it possible that Bob is not rational, Bob could possibly choose the payoffs $A:7$, $E:1$, which then would maximize Anns payoff. Therefore the common knowledge of rationality is needed.


\begin{figure}[htbp]
\centering
	\begin{tikzpicture}[every node/.style={draw,circle}, align=center, node distance=1.9cm]
	%Nodes
	\node[draw,fill=black]   	(q1)   {};
	\node[draw, below right=1cm of q1,style={draw=none}]      (q2)   {$_{E:2}^{A:3}$};
	\node[draw, below left of=q1]     (q3)   {};
	\node[draw,fill=none, below right of=q3,style={draw=none}]    (q4)   {$_{E:4}^{A:2}$};
	\node[draw,fill=none, below left of=q3,style={draw=none}]      (q5)   {$_{E:1}^{A:7}$};
	íges
	\draw [line] (q1) -- (q2) node [label={[shift={(-0.6,-0.5)}]2},very near start, right, draw=none, fill=none] (TextNode) {1};
	\draw [line] (q1) -- (q3) node [very near start, left, draw=none, fill=none] (TextNode) {2};
	\draw [line] (q3) -- (q4) node [very near start, right, draw=none, fill=none] (TextNode) {1};
	\draw [line] (q3) -- (q5) node [very near start, left, draw=none, fill=none] (TextNode) {2};
	\end{tikzpicture}
	\caption{Game tree for a perfect-information game between Ann and Bob.}
	\label{prg:lec5}
\end{figure}


Here, it can be seen that common knowledge of rationality implies backward induction. To proof that this is actually the case, one must look into epistemic logic to try and model the notion of common knowledge of rationality. 

First step is to create a possible worlds model $M$, where $\Omega$ is the set of all states of the world. 
Let $s$  be a function defined by
$$
s(\omega) = \times_i S_i
$$
where $S_i$ is the set of all strategies of player $i$. The function $s$ is therefore a function that maps a state $\omega$ in the world model, to a tuple of the players strategies at $\omega$.
Let the set of states in $M$, where $s(\omega)$ returns strategies that are rational for each players in the game, be denoted $R$. Let the set of states where $R$ is common knowledge be denoted $CR$. At some states in the $M$, the strategy represented in the state leads to the inductive outcome. Let $I$ denote these states.
In \cite{aumann1995a} a detailed proof is given to show that:
$$
CR \subset I.
$$
Which means that common knowledge of rationality implies correctness of backward induction.

\subsection{Discussion}
It is worth to investigate a play of a game where some nodes are not reached. Previously we assumed that a player would act rationally at every point in the game, but let's assume that we only demand rationality at the nodes that are actually reached in the game tree.
If a node is never reached, then one could possibly say that any decision made in that node is actually rational, but it's obvious that this prevents backward induction from working correctly. Therefore it is important to assume rationality at all nodes.

Common knowledge of rationality is a quite rare scenario in practice, given that the players are human. Often human players are not rational at all or at least it is not common knowledge. In these cases it would be irrational to choose a strategy based on the result of the backwards induction, because a player do not know the reasons for his/hers opponents choices.

From a human perspective, perfect-information games with common knowledge of rationality are not fun to play, as the course of the game is completely known before the first move is made. It would be a complete waste of time to play a game and reach an outcome that was determined beforehand.
