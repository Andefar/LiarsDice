In games of imperfect information, the players are not completely sure of the current state of the game. That is, one or more players cannot tell two or more states apart. This is a classic application of a possible worlds model. In this section, based on \cite{benthem2001a}, we are going to combine game trees and possible worlds models in order to model games of imperfect information\dots

{ \color{red} I don't think bisimulations are important in this article. This is a introductory article on the connections between game theory and epistemic logic. Bisimulations are a more advanced tool for reducing the size of your games so that they are easier to handle. Hence, they are mostly a tool for gaining efficiency and efficiency is not in introductory topic. I would rather describe how to compute the Nash equilibrium of the Liar's dice example. }

\subsection{Perfect information game trees}

{ \color{red} We need this subsection so that we can contrast with games of imperfect information in later subsections. We only need to describe the \emph{the dynamic modal language}, if we want to write down concrete strategies for our example. }

\begin{itemize} \color{red}
\item Formal definition of perfect-information game trees
\item Formal definition of strategies in perfect-information games.
\item Short introduction to outcomes and powers (C1, C2, and C3).
\end{itemize}

\subsection{Imperfect information game trees}

\begin{itemize} \color{red}
\item Formal definition of imperfect-information game trees
\item Strategies in imperfect games (the discussion from V.3). Maybe it is better to describe this after perfect recall.
\item Short introduction to outcomes and powers in imperfect games (C3 does not hold anymore).
\item Remarks on the looseness of this definition: "Players need not know what the opponent has played, or what they played themselves, they need not know if it is their turn, or whether the game has ended, etc. One can think up plausible scenarios with all of the pictures shown in Figure 6."
\item How to restrict the imperfect-information-game-trees model to be more "realistic": Describe the examples from "III.3. Constraints for special axioms".
\end{itemize}

\subsection{Liar's Dice}

As an example of a game of imperfect information, we now consider the game Liar's Dice. In this game\footnote{A variety of dice games are called Liar's Dice. They all have elements of concealed die rolls and deception. The variant we consider is probably the simplest.}, two players take turns rolling a single die under a cup, privately looking at the result, and then call out a possible result of the die roll. The player may lie about the outcome of the roll. Now, the opponent can accept the call and start a new round by re-rolling the die or she can challenge the call by lifting the cup. If the call was correct, she looses the game and if the call was a lie, she wins the game. In each round, a player must call out a greater result than the one called in the previous round.

This game is zero-sum, imperfect, \dots

{ \color{red} Here, we need a formal definition of Liar's Dice or the game tree (for a dice of only two faces) or both. }

\subsection{Perfect recall}

{ \color{red} We need to note that because of the way we model Liar's Dice, the players have perfect recall. We also need to note that perfect recall is not needed to play Liar's Dice perfectly. }

{ \color{red} Predictive strategies? Ask Nina\dots }

\subsection{Information update}

{ \color{red} This is the least important and should be left out, if we run out of space. }